\subsection{Coloumb Law}
\begin{equation}
	F_e = \frac{q_1 q_1}{4\pi \epsilon_0 r^2} \qquad[C] \tag{Forza di Coloumb}
\end{equation}
Dove $\epsilon_0$ è la permittività elettrica.

\subsection{Flusso Elettrico}
\paragraph{Flusso elettrico su una superficie}
\begin{equation}
	\Phi_E = \int_{S} \vec{E} \cdot \ud \vec{A}
\end{equation}

\subsection{Legge di Gauss}
Sia S una \textbf{Superficie chiusa} e $q_{in}$ la carica interna alla superficie allora:
\begin{equation}
	\Phi_E = \oint_{S} \vec{E} \cdot \ud \vec{A} = \frac{q_{in}}{\epsilon_0}
\end{equation}
Cioè il flusso dipende solo dalle sorgenti di campo contenute nella superficie.
\subsection{Legge di Ampere}
Descrive i campi magnetici creati facendo passare corrente attraverso (per esempio) ad un cavo. La direzione del campo magnetico si ottiene con la regola della mano destra.

