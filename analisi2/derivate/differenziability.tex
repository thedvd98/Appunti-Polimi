\subsection{Differenziabilità}

\subsubsection{Funzione differenziabile} Una funzione è differenziabile nel punto $(0,0)$ se può essere approssimata nelle vicinanze di questo punto dalla seguente funzione lineare:
\begin{equation}
	f(x+h, y+k) = f_x(x,y)h + f_y(x,y)k + f(x,y) + \varepsilon \sqrt{h^2 + k^2}
\end{equation}
dove $\varepsilon, h, k \rightarrow 0$ e $\sqrt{h^2 + k^2}$ denota la distanza tra il punto $(x+h,x+k) \text{ e } (x,y)$
Le funzioni differenziabili hanno le derivate parziali e anche le derivate in ogni direzione.

\paragraph{Il Piano tangente} nel punto $(\xi, \eta)$:
\begin{equation}
	z(x,y) = f(\xi, \eta) + f_{x}(x - \xi) + f_{y}(y-\eta)
\end{equation}

\subsubsection{Derivate direzionali}
\begin{align*}
	D_{\theta}f(x,y) & = \lim_{\rho \rightarrow 0} \frac{f(x+\rho \cos\theta, y+\rho\sin\theta) - f(x,y)}{\rho} \\
			   & = f_x \cos(\theta) + f_y \sin(\theta)
\end{align*}
\paragraph{Gradiente}: $\vec{\nabla f} = \Big(f_x(x,y), f_y(x,y)\Big)$, esso è il vettore che indica la direzione della massima pendenza sul grafico ed è sempre perpendicolare alle curve di livello.
\begin{align*}
	D_{\hat{v}}f(x,y) = \vec{\nabla f} \cdot \hat{v}
\end{align*}
Dove $\hat{v}$ è un versore.

