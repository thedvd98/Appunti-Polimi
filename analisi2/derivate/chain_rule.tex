\subsection{Chain Rule}

Sia $u = f(\xi, \eta, ...)$ una funzione con argomenti che sono loro stessi funzioni.
\begin{align*}
	\xi = \Phi(x,y) \\
	\eta = \Psi(x,y)
\end{align*}
Le sue derivate parziali saranno date da:
\begin{align*}
	u_x = f_\xi \xi_x + f_\eta \eta_x + . . . \\
	u_y = f_\xi \xi_y + f_\eta \eta_y + . . .
\end{align*}

\paragraph{Esempio:}
Derivata di $x^x$
\begin{align*}
	[u = x, v = x, z = u^v] \\
	z_x = z_u u_x + z_v v_x = vu^{v-1} + u^v\log(u) \\
	= xx^{x-1} + x^x\log(x)
\end{align*}



