\subsection{Integrali Multipli}
\subsubsection{Trasformazioni}
\paragraph{Matrice Jacobiana} è la matrice delle derivate parziali delle funzioni che vengono sostituite.\\
$x = f(u,v)$\\
$y = g(u,v)$

Area Parallelogramma
\begin{math}
\displaystyle {
\begin{aligned}\mathbf {a} \times \mathbf {b} = \det{
	\begin{bmatrix}
		a_{1} & b_{1}\\
		a_{2} & b_{2}
	\end{bmatrix}
	}
\end{aligned}
}
\end{math}\\

\begin{align}\mathbf {D} = \det{
	\begin{bmatrix}
		\dfrac{\partial f}{\partial u} & \dfrac{\partial g}{\partial u}\\[1em]
		\dfrac{\partial f}{\partial v} & \dfrac{\partial g}{\partial v}
	\end{bmatrix}
	}
	\tag{determinante matrice Jacobiana}
\end{align}


\begin{equation}
	\int\int_{D} f(u,v) g(u,v)|D|\ud u \ud v
\end{equation}
Nel caso delle \textbf{trasformazioni polari} $|D|$ sarà sempre uguale a $\rho$.

\paragraph{Esempio coordinate ellittiche} un esempio di base: \\
$\iint_{D} \sqrt{4 x^{2}+9 y^{2}} d x d y$ \\
$\mathrm{D}=\left\{(x, y): x \geq 0 ; y \geq-\frac{2}{3} x ; 4 x^{2}+9 y^{2} \leq 4\right\} \\
\frac{x^{2}}{a^{2}}+\frac{y^{2}}{b^{2}}=\rho^{2}\\
x=a \rho \cos \theta \\
y=b \rho \sin \theta \\
|\operatorname{det} J|=a b \rho \\
\text { nel nostro caso } a=1 / 2, b=1 / 3 \text { e } \rho=2 \\
\text { quindi le nuove coordinate sono } \\
x= \rho \frac{1}{2}\cos\theta\\
y=\rho \frac{1}{3} \sin \theta \\
|\operatorname{det} J|=\frac{1}{6} \rho \\
$
Il $\theta$ da cui partire si trova dall'equazione $y= -\frac{2}{3}x \text{ che diventa in coordinate polari}\\
\frac{1}{3} \rho \sin\theta = -\frac{2}{3} \rho\frac{1}{2}\cos\theta\quad$ da cui trovo che $\theta = \frac{\pi}{4}$.\\
E quindi l'integrale finale risulterà essere:
\begin{equation*}
	\int_{0}^{2} \int_{-\frac{\pi}{4}}^{\frac{\pi}{2}} \frac{1}{6} \rho \rho \ud\rho \ud\theta
\end{equation*}
