\subsection{Integrali di linea}

\subsubsection{Linee regolari}
Generica 
\begin{equation}
	P(t) = 0 + x(t) \hat{i} + y(t) \hat{j} + z(t) \hat{k}
\end{equation}
\paragraph{Una linea è regolare se rispetta le seguenti condizioni:}
\begin{itemize}
	\item \textbf{Semplice} quando non esistono $t_1 \neq t_2: P(t_1) = P(t_2)$
	\item $P(t) \in C^1 ([a,b])$
	\item $P^{'} (t) \neq 0\quad \forall t \in (a,b)$
\end{itemize}

\subsubsection{Lunghezza linee}
\begin{itemize}
	
	\item $y = f(x)$
\begin{equation}
	l = \int_{a}^{b} \sqrt{1+[f^{'}(x)]^2}\ud x
\end{equation}
	\item $x=\varphi(t), y = \psi(t), z = \tau(t)$
\begin{equation}
	l = \int_{a}^{b} \sqrt{\varphi^{'}(t)^2 + \psi^{'}(t)^2 + \tau^{'}(t)^2} \ud t
\end{equation}
\item $\rho = \rho(\theta) \quad \theta \in [\alpha,\beta]$
\begin{equation}
	l = \int_{\alpha}^{\beta} \sqrt{\rho^2 + \rho^{'}^2} \ud \theta
\end{equation}
\end{itemize}

\subsubsection{Integrali di superficie}
\begin{equation*}
	A = \int\int_{R} \sqrt{1+f_x ^2 + f_y ^2} \,\ud R	
\end{equation*}
Altra formula:
\begin{equation*}
	A = \int\int_{R} \sqrt{f_x ^2 + f_y ^2 + f_z ^2} \abs{\frac{1}{f_z}} \,\ud x \ud y
\end{equation*}

\subsubsection{Integrali su campi vettoriali}
\paragraph{Lavoro}
Sia $\mathbf{F} = P(x,y)\hat{i} + Q(x,y)\hat{j}$ un campo vettoriale (per esempio una forza che cambia a seconda della posizione x,y).\\
Mentre $\ud \mathbf{S} = \ud x \hat{i} + \ud y \hat{j}$ cioè il vettore tangente alla curva.\\
L'integrale di linea di F lungo la linea C è dato da:
\begin{equation*}
	\int_{C} \mathbf{F} \cdot \ud\mathbf{S} = \int_{C} P(x,y)\ud x + Q(x,y) \ud y
\end{equation*}

\paragraph{Funzione Potenziale}
Una $f(x,y)$ è una funzione potenziale di un campo vettoriale $F = P\hat{i} + Q\hat{j}$ se il gradiente di f è $P\hat{i} + Q\hat{j}$. \\
Non tutti i campi vettoriali hanno una funzione potenziale.\\

\begin{teorema}[Teorema per la verifica dell'esistenza di una funzione potenziale]
	Un campo vettoriale $P i+ Qj$ ha una funzione potenziale se e solo se
	\begin{equation*}
		\frac{\partial P}{\partial y} = \frac{\partial Q}{\partial x}
	\end{equation*}
\end{teorema}
Un \textbf{differenziale} è \textbf{esatto} se vale la qua sopra citata formula.
\begin{teorema}[Percorso indipendente]
	Sia f una funzione potenziale del campo vettoriale $Pi+Qj$
	\begin{equation}
		\int_{C} P \ud x + Q \ud y = f(B)-f(A) = \int_{A}^{B} P\ud x + Q\ud y
	\end{equation}
	Permette di trovare anche la funzione potenziale dal campo vettoriale.
	Sia g una funzione potenziale se e solo se g si può scrivere nel seguente modo:
	\begin{equation*}
		g(x,y) = \int_{A}^{(x,y)}P\ud x + Q \ud y + K
	\end{equation*}
\end{teorema}
Un Campo vettoriale con una funzione potenziale viene detto \textbf{Campo conservativo}.\\
