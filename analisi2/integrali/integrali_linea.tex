\subsection{Integrali di linea}

\subsubsection{Linee regolari}
Generica 
\begin{equation}
	P(t) = 0 + x(t) \hat{i} + y(t) \hat{j} + z(t) \hat{k}
\end{equation}
\paragraph{Una linea è regolare se rispetta le seguenti condizioni:}
\begin{itemize}
	\item \textbf{Semplice} quando non esistono $t_1 \neq t_2: P(t_1) = P(t_2)$
	\item $P(t) \in C^1 ([a,b])$
	\item $P^{'} (t) \neq 0\quad \forall t \in (a,b)$
\end{itemize}

\subsubsection{Lunghezza linee}
\begin{itemize}
	
	\item $y = f(x)$
\begin{equation}
	l = \int_{a}^{b} \sqrt{1+[f^{'}(x)]^2}\ud x
\end{equation}
	\item $x=\varphi(t), y = \psi(t), z = \tau(t)$
\begin{equation}
	l = \int_{a}^{b} \sqrt{\varphi^{'}(t)^2 + \psi^{'}(t)^2 + \tau^{'}(t)^2} \ud t
\end{equation}
\item $\rho = \rho(\theta) \quad \theta \in [\alpha,\beta]$
\begin{equation}
	l = \int_{\alpha}^{\beta} \sqrt{\rho^2 + \rho^{'}^2} \ud \theta
\end{equation}
\end{itemize}

\subsubsection{Integrali di superficie}
\begin{equation*}
	A = \int\int_{R} \sqrt{1+f_x ^2 + f_y ^2} \,\ud R	
\end{equation*}
Altra formula:
\begin{equation*}
	A = \int\int_{R} \sqrt{f_x ^2 + f_y ^2 + f_z ^2} \abs{\frac{1}{f_z}} \,\ud x \ud y
\end{equation*}
