\documentclass[a4paper]{article}
\usepackage[utf8]{inputenc}
\usepackage{amsmath}
\usepackage{mathtools}
\usepackage{amsthm}
\usepackage{amsfonts} %per mathfrak
\usepackage{hyperref}

% Tuttta sta roba per far apparire degli abs decenti
\DeclarePairedDelimiter\abs{\lvert}{\rvert}%
\makeatletter
\let\oldabs\abs
\def\abs{\@ifstar{\oldabs}{\oldabs*}}

\newcommand{\ud}{\,\mathrm{d}}
\author{Davide}

% Commento
\usepackage{circuitikzgit}
\usepackage{empheq}

\title{Elettrotecnica}

\begin{document}
\maketitle
\section{Cenni di elettromagnetismo}
\subsection{Coloumb Law}
\begin{equation}
	F_e = \frac{q_1 q_1}{4\pi \epsilon_0 r^2} \qquad[C] \tag{Forza di Coloumb}
\end{equation}
Dove $\epsilon_0$ è la permittività elettrica.

\subsection{Flusso Elettrico}
\paragraph{Flusso elettrico su una superficie}
\begin{equation}
	\Phi_E = \int_{S} \vec{E} \cdot \ud \vec{A}
\end{equation}

\subsection{Legge di Gauss}
Sia S una \textbf{Superficie chiusa} e $q_{in}$ la carica interna alla superficie allora:
\begin{equation}
	\Phi_E = \oint_{S} \vec{E} \cdot \ud \vec{A} = \frac{q_in}{\epsilon_0}
\end{equation}
Cioè il flusso dipende solo dalle sorgenti di campo contenute nella superficie.
\subsection{Legge di Ampere}
Descrive i campi magnetici creati facendo passare corrente attraverso (per esempio) ad un cavo. La direzione del campo magnetico si ottiene con la regola della mano destra.


\section{Legge di Ohm}
Legge di Ohm:
\[V = RI\]
\[R = \Omega \qquad I = [A] \qquad V = [V] \]

\section{LKT}
\[\sum_{k=0}^{n} V_k = 0\]
\section{LKC}
\[\sum_{k=0}^{n} I_k = 0\]

\section{Tripoli}
	Circuito a stella:
\begin{center}
	\begin{circuitikz}
		\ctikzset{label/align = straight}
		\draw
		(0,0) to[R, l=\(R_a\), i>^=\(i_1\)] (2,-2) {}
		(2,-2) to[R, l=\(R_b\), i^<=\(i_2\)] (4,0)
		(2, -2) to[R, l=\(R_c\), i=\(i_3\)] (2, -4.5)
	;\end{circuitikz}
\end{center}
	Triangolo di resistori:
\begin{center}
	\begin{circuitikz}
		\ctikzset{label/align = straight}
		\draw
		(-2,0) to[short, i>^=\(i_1\)] (0,0)
		(0,0) to[R, l=\(R_2\), i>^=\(i_c\), *-*] (4,0)
		(4,0) to[short, i^<=\(i_2\)] (6,0)
		(4,0) to[R, a=\(R_3\), i=\(i_a\), *-*] (2,-3)
		(0, 0) to[R, l=\(R_1\), i=\(i_b\), *-*] (2, -3)
		(2,-3) to[short, i=\(i_3\)] (2,-5)
	;\end{circuitikz}
\end{center}
Trasformazioni:

Circuito a triangolo \(\rightarrow\) stella
\begin{empheq}[box=\fbox]{align*}
R_a = \frac{R_1R_2}{R_1+R_2+R_3}\\
\\
R_b = \frac{R_2R_3}{R_1+R_2+R_3}\\
\\
R_c = \frac{R_1R_3}{R_1+R_2+R_3}
\end{empheq}
Circuito a stella \(\rightarrow\) triangolo 
\begin{empheq}[box=\fbox]{align*}
R_1 = R_a + R_b + \frac{R_aR_b}{R_c}\\
\\
R_2 = R_a + R_c + \frac{R_aR_c}{R_b}\\
\\
R_3 = R_c + R_b + \frac{R_cR_b}{R_a}
\end{empheq}
\section{Amplificatori operazionali}
\begin{center}
	\begin{circuitikz} \draw
		(0,0) node[op amp] (opamp) {}
		(opamp.+) node[left] {$v_+$}
		 (opamp.-) node[left] {$v_-$}
		 (opamp.out) node[right] {$v_o$}
		 (opamp.down) --++(0,-0.5) node[vee]{}
	;\end{circuitikz}
\end{center}

\(V_0\)\qquad Tensione tra terminale di uscita e terra
\(V_d = V_{in} = V_+ - V_-\)

Nell' OpAmp Ideale le correnti d'ingresso sono nulle e \(V_d = 0\)

\end{document}
