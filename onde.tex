\documentclass[16pt]{article}
\usepackage[utf8]{inputenc}
\usepackage{amsmath}
\usepackage{amsthm}

%\newcommand{\ud}{\,\mathrm{d}}

\title{Onde Elettromagnetiche e Mezzi Trasmissivi}

\begin{document}
\maketitle

\section{Basi}

\paragraph{Banda del Segnale} $\mathbf{B_s}$ insieme delle frequenze contenute nel segnale.
\paragraph{Banda Passante} $\mathbf{B_m}$ insieme di frequenze che soddisfano una certa qualità.

\subsubsection{Attenuazione}
L'ampiezza di un onda elettromagnetica viene attenuata in maniera di tipo esponenziale $e^{-\alpha z}$.
\paragraph{Attenuazione} $\alpha [Np/m]$
\paragraph{Fattore di conversione} per ottenere $\alpha$ in [dB/m] avendo $\alpha$ in [Np/m]
\begin{align*}
	\alpha_{dB}z &= -10\log_{10}e^{-2\alpha z} \\
		     & = \mathbf{8.68\, \alpha\,z} \\
	\alpha_{dB} &=\mathbf{8.68\, \alpha}
\end{align*}
\subsubsection{Velocità}
\paragraph{Costante dielettrica relativa:}  $\mathbf{\varepsilon_r}$ dipende dal mezzo di trasmissione
\paragraph{Indice di rifrazione:} $\mathbf{n=\sqrt{\varepsilon_r}}$
\begin{equation}
	v_f = \frac{c}{\sqrt{\varepsilon_r}} = \frac{c}{n}
\end{equation}

\end{document}
