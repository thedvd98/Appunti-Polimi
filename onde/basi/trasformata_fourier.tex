\subsection{Trasformata di Fourier}
\textbf{L'antitrasformata di Fourier}
\begin{equation}
	s(t) = \mathcal{F}^{-1}\{S(f)\}= \int_{-\infty}^{\infty}S(f)\,e^{j\,2\pi ft}\ud f
\end{equation}
$S(f)$ viene chimata \textbf{trasformata di Fourier}
\begin{equation}
	S(f) = \mathcal{F} \{s(t)\} = \int_{-\infty}^{\infty} s(t)e^{-j2\pi ft} \ud t
\end{equation}
\paragraph{Segnale ritardato} Sia s(t) un segnale, il segnale ritardato $z(t) = s(t-\tau)$ ha trasformata
\begin{equation}\label{trasformata_segnale_ritardato}
	Z(f) = \mathcal{F}\{s(t-\tau)\} = S(f)e^{-j2\pi f\tau}
\end{equation}

